% ! TeX root = ..\main.tex
\lecture{2}{Lecture 1}{Tue 4 Oct 2021 14:00}{}

\subsection{Discrete Tools}

Let $A$ be an event in a $\sigma$-algebra $\mathcal F$, and let $P(A)=\frac{|A|}{|\Omega|}$ be a probability measure. Note that if we can break the experiment we are interested in into $k$ subexperiments $\Omega_i\subseteq \Omega$, then the multiplication rule dictates

$$
|\Omega|=|\Omega_1|\times |\Omega_2|\times \cdots |\Omega_k|.
$$
\subsubsection{Permutations}

\begin{definition}
Take $n$ distinct objects, and choose $k$ of them to be put \textit{in a specific order.} A \textbf{permutation} refers to one such ordering.
\end{definition}
We can find the number of possible permutations of size $k$ using the multiplication rule:

\begin{align*}
    \underbrace{n}_{\text{1st choice}}\times \underbrace{(n-1)}_{\text{2nd}}\times \cdots \times \underbrace{(n-k+1)}_{k\text{th}}&=\frac{n(n-1)\cdots 1}{(n-k)(n-k-1)\cdots 1} \\
    &=\frac{n!}{(n-k)!}\\
    &= \ ^nP_k.
\end{align*}

\subsubsection{Combinations}
\begin{definition}
Take $n$ distinct objects, and choose $k$ of them, but do \textit{not} put them in order. A \textbf{combination} refers to one such group. The number of combinations of size $k$ is represented as $^n C_k$
\end{definition}
Note that a permutation can be represented as

$$
\underbrace{\binom{\text{choose $k$ objects}}{\text{out of $n$}}}_{\ ^nC_k} \times \underbrace{\binom{\text{put these $k$}}{\text{objects in order}}}_{k!}.
$$

Hence,

$$
^nP_k=^nC_k\times k! \implies ^nC_k=^nP_k=\frac{n!}{(n-k)!k!}.
$$
\begin{notation}
We also denote combinations by $\binom{n}{k},$ also referred to as the \textbf{binomial coefficient.}
\end{notation}
\vskip.1cm
In general, 

$$
(a+b)^n=\sum^n \binom{n}{j}a^jb^{n-j}.
$$

But why?
\begin{remark}
Take $n$ objects, $k$ of type I, and $n-k$ of type II. Put these $n$ objects in order. How many possible ways of ordering them? We have $\binom{n}{k}.$ Again, why? Think of it this way. Suppose there are $n$ slots. We can put an object of type I or II in each slot. The order doesn't matter, so there are $\binom{n}{k}$ ways to choose $k$ slots. 
\end{remark}

But how does this relate to the binomial coefficient? First note that

$$
(a+b)^n=\underbrace{(a+b)(a+b)(a+b)\cdots (a+b)}_{n \text{ times}}
$$

Now consider the form of each term of the polynomial:
$$
a^jb^{n-j}
$$

Each term can be thought of as one combination of slots. We "choose" $a$ or $b$ for each part of the product, and multiply them together to get a term of the form $a^kb^{n-k}.$ By the above, $\binom{n}{k}$ is how many terms are constructed for each $k.$

\begin{eg}
	Let $\{1,2,\ldots,59\}$ be a set of numbers. Choose 6 without replacement. What is the probability that I match all 6 if I draw at random? Two ways to solve this:
	\begin{enumerate}
	\item \underline{Order Matters}: suppose that we consider every permutation of drawings. Then
			\[
					|\Omega|={}^{59} P_6 = \frac{59!}{53!}
			.\] 
			Now let $A$ be the event that we match all 6 numbers, regardless of order. Then $|A|=6!$, and
			\[
			P(A)=\frac{|A|}{|\Omega|}=\frac{6!\,53!}{59!}
			.\] 
	\item \underline{Order Doesn't Matter}: suppose that we consider ever \textit{combination} of drawings. Then
			\[
					|\Omega|={}^{59}C_6=\frac{59!}{53!\,6!}
			.\] 
	\end{enumerate}
	Let $A$ be the event that we match all 6 numbers. Then $|A|=1,$ because there is only one combination that fits the criteria of $A$. Then
	\[
			P(A)=\frac{1}{{}^{59}C_6}=\frac{6!\,53!}{59!}
	,\]
	the same answer as before.
\end{eg}
\subsection{Conditional Probability}

Let $(\Omega, ]\mathcal F, P)$ be a probability space. Let $B\in \mathcal F$ with $P(B)>0.$ Define a new probability measure $P_B$ such that $A\in \mathcal F,$ and

$$
P_B(A)=P(A\mid B)=\frac{P(A\cap B)}{P(B)}.
$$

Note that if $P(A)=\frac{|A|}{|\Omega|},$ then 

$$P(A\mid B)=\frac{|A\cap B|/|\Omega|}{|B|/|\Omega|}=\frac{|A\cap B|}{|B|}.$$

So, 
$$
P(A^c\mid B)=1-P(A\mid B).
$$

If $A_1,A_2,\dots \in \mathcal F,$ and $P(A_i)>0$ for $P(A_i)>0$ for $i=1,2,\dots,$ then

$$
P(A_n\cap \cdots \cap A_1)=P(A_n\mid A_{n-1}\cap \cdots \cap A_1)P(A_{n-1}\mid A_{n-2} \cap \cdots A_1)\cdots P(A_1)
$$

\subsection{Bayes' Rule}

Let $\mathcal F$ be a $\sigma$-algebra. For two events $A,B\in\mathcal F$,

\begin{align*}
    P(A\mid B) &= \frac{P(A\cap B)}{P(B)} \\
    &=\frac{P(A\cap B)}{P(B)} \\
    &= \frac{P(B\mid A)P(A)}{P(B)} \\
    &= P(A)\frac{P(B\mid A)}{P(B)}.
\end{align*}

We refer to $P(A)$ as the \textbf{prior} and $P(B\mid A)$ as the Bayes factor.

\subsection{The Law of Total Probability}
\label{ssec:TLoTP}

\begin{definition}
A partition of $\Omega$ is a collection of events $\{B_1, B_2, \dots\}$ such that
\begin{enumerate}[i.]
    \item $P(B_i)>0$ for all $i$,
    \item $\bigcup^\infty_{i=1}=\Omega$ (collectively exhaustive),
    \item $B_i\cap B_j=\varnothing$ for all $i\neq j$ (pairwise mutually exclusive).
\end{enumerate}
\end{definition}
\begin{theorem}[Law of Total Probability]
Let the set $\{B_1,B_2,\dots\}$ be a partition of $\Omega.$ Then for any $A\in \mathcal F,$

$$P(A)=\sum^\infty_{i=1}P(A\cap B_i)=\sum^\infty_{i=1}P(A\mid B_i)P(B_i).$$
\end{theorem}
