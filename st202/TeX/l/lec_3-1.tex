% ! TeX root = ..\main.tex
\lecture{3}{Lecture 1}{Tue 12 Oct 2021 14:00}{}

\subsection{Examples of Random Variables}

\begin{eg}
	Let $X$ be a random variable. For $x=2,$ we have $A_2\in \mathcal F,$ so we can write $P(A_2)=P(X\leq 2).$
\end{eg}


\begin{eg}
Suppose there is a family with two children. Let $X=$ the number of girls. Then
\begin{align*}
	&P(A_0)=P(\{BB\})=\frac{1}{4} \\
	&P(A_1)=P(\{BB,BG,GB\})=\frac{3}{4} \\
	&P(A_{\frac32})=P(A_1)=\frac34 \\
	&P(A_2)=P(\Omega)=1 \\
	&P(A_{-1})=P(\{\})=0 \\
	&P(A_\pi)=P(\Omega)=1.
\end{align*}
\end{eg}
\subsection{The Cumulative Distribution Function}
\vskip.1cm
\begin{definition}
		A random variable $X$ is \textbf{positive} if $X(\omega)\geq 0$ for all $\omega\in \Omega.$
\end{definition}

\begin{definition}
		The \textbf{cumulative distribution function} (\textbf{CDF}) of a random variable $X$ is the function $F_X:\mathbb R\to [0,1]$ given by $F_X(x)=P(\underbrace{X\leq x}_{A_x}).$
\end{definition}

\begin{eg}
Let $X$ be a random variable and $F_X$ be a valid CDF. Then
$$P(A_1)=P(X\leq 1)=F_X(1).$$ 
\end{eg}

\begin{eg}
In our two child example,
$$
F_X(0)=\frac{1}{4}, \quad F_X(1)=\frac34, \quad F_X(2)=1, \quad F_X(-1)=0, \quad F_X\left(\frac32\right)=\frac34, \dots
$$
Moreover, note that the CDF in this case is a step function, as seen in the figure below.
\incfig[0.7]{w3_l1_fig1}{The Cumulative Distribution Function of the two child example.}
\end{eg}

\begin{definition}
A function $g:\mathbb R\to \mathbb R$ is \textbf{right-continuous} if $g(x+)=g(x)$ for all $x\in \RR$, where
\begin{align*}
     &g(x+)=\lim_{h\downarrow 0}g(x+h), \\
    \text{and} \quad &g(x-)=\lim_{h\downarrow 0}g(x-h).
\end{align*}
\end{definition}

\begin{prop}
If $F_X$ is a CDF, then 
\begin{enumerate}[i.]
    \item $F_X$ is increasing, i.e., if $x<y$ then $F_X(x)\leq F_X(y).$
    \item $F_X$ is right-continuous, i.e. $F_X(x+)=F_X(x)$ for all $x\in \RR.$
    \item $\lim_{x\to -\infty}=0$ and $\lim_{x\to \infty} F_X(x)=1$.
\end{enumerate}
\end{prop}

\begin{proof}
\begin{enumerate}[i.]
\item
If $x<y,$ then $A_X\subseteq A_y$, so
$$
F_X(x)=P(A_x)\leq P(A_y)=F_X(y).
$$
\item
Take a decreasing sequence $\{x_n\}$ such that $x_n\downarrow x$ as $n\to \infty$ $(x_1\geq x_2\geq x_3\geq \cdots).$ We have
$$
A_{x_1}\supseteq A_{X_2} \supseteq \cdots
$$
and $A_x\supseteq A_{x_n}$. So 
$$
A_x=\bigcap^\infty_{n=1}A_{x_n}.
$$
Then
\begin{align*}
    \lim_{x\to \infty} F_X(x_n)&=\lim_{n\to \infty}P(A_{x_n})\\
    &=P\left(\bigcap_{n\in\mathbb N}A_{x_n}\right)\\
    &=P(A_x) \\
    &=F_X(X) \\
	\implies \lim_{h\downarrow 0} F_X(x+h)&=F_X(x).
\end{align*}

\item In M\&P textbook.
\end{enumerate}

\end{proof}
\subsubsection{Some basic properties of CDFs}
Observe that
\begin{itemize}
    \item $P(X>x)=1-P(X\leq x)=1-F_X(x)$
    \item $P(x<X\leq y)=F_X(y)-F_X(x)$
    \item $P(X<x)=F_X(x-)$
    \item $P(X=x)=F_X(x)-F_X(x-).$
\end{itemize}
