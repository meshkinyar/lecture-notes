% ! TeX root = ..\main.tex

\chapter{Multivariate Distributions}

\lecture{8}{Lecture 1}{Tue 16 Nov 2021 14:00}{}
\subsection{Joint CDFs and PDFs}

\begin{recall}
Note that
$$
F_X: \RR\to [0,1] \quad F_{X_1,\dots,X_n}: \RR^n\to [0,1].
$$
\end{recall}

\begin{definition}
		The \textbf{joint cumulative distribution function} of \phantom{hello again!} $X_1,\dots,X_n$ is the function $$F_{X_1,\dots,X_n}(x_1,\dots,x_n)=P(X_1\leq x_1,X_2\leq x_2, \ \dots, \ X_n\leq x_n).$$
\end{definition}

Note that the commas in the last expression indicate $\cap.$

\subsubsection{Bivariate CDFs}

\begin{notation}
We write a bivariate CDF as
$$F_{X,Y}(x,y)=P(X\leq x, Y\leq y).$$
\end{notation}

Note that
\begin{align*}
		P(x_1<X\leq x_2,\ y_1<Y\leq y_2)&=&F_{X,Y}(x_2,y_2)-F_{X,Y}(x_1,y_2) \\
		& &-F_{X,Y}(x_2,y_1)+F_{X,Y}(x_1,y_1).
\end{align*}
Moreover,
$$
F_{X,Y}(-\infty,y)=0=F_{X,Y}(x,-\infty)
$$
Similarly,
$$
F_{X,Y}(\infty,\infty)=1.
$$

Lastly,
\begin{align*}
  F_{X,Y}(x,\infty)&=\lim_{y\to \infty}F_{X,Y}(x,y)\\
  &=P(X\leq x, Y\leq \infty)\\
  &=P(X\leq x)\\
  &=F_X(x),
\end{align*}
which is defined as the marginal CDF of $X.$ Naturally,
$$
\lim_{x\to \infty} \FXY=F_Y(y).
$$
\vskip.1in
If $X,Y$ are both discrete, the joint PMF is $\fXY=P(X=x,Y=y).$ So
$$
\FXY=\sum_{u\leq x}\sum_{v\leq y}f_{X,Y}(u,v).
$$

\begin{eg}
Draw 2 cards from a deck of 52 cards. Let $X:$ number of kings drawn, and $Y:$ the number of aces drawn. Note that
$$
f_{X,Y}(0,0)=\frac{44}{52}\frac{43}{51}\approx 0.713.
$$
We can represent the probabilities of each event using an array:
\begin{figure}[H]
$$
\begin{array}{c|c|c|c|c}
		x\downarrow y \rightarrow & 0 & 1 & 2 & f_X(x) \\
		\hline
		0 & 0.713 & 0.133 & 0.004 & 0.850 \\
		1 & 0.133 & 0.012 & 0 & 0.145 \\
		2 & 0.004 & 0 & 0 & 0.004 \\
		\hline
		f_Y(y) & 0.85 & 0.145 & 0.004 & 1
\end{array}
$$
\caption{An array representing the probabilities for values of $x$ and $y$.}
\label{fig:card_table}
\end{figure}

It follows that
$$
\sum_x\sum_y \fXY=1.
$$
and that
$$
f_X(x)=\sum_y \fXY.
$$
\end{eg}
\vskip.1in

\begin{definition}
Random variables $X,Y$ are jointly continuous if
$$
\FXY=\int^y_{-\infty}\int^x_{-\infty} \fXY(u,v) \ du \ dv
$$
for all $x,y\in \RR.$
\end{definition}
So
$$
\fXY=\frac{\partial^2}{\partial x \partial y}\FXY.
$$

Now, we have

$$
\int_{\RR^2} \fXY \ dx \ dy=1,
$$
and
$$
f_X(x)=\intR \fXY \ dy, \quad \quad  f_Y(y)=\intR \fXY \ dx,
$$
and
$$
P((X,Y)\in B)=\int\int_B \fXY \ dx \ dy.
$$
\begin{remark}
Aside: Note that
$$f_X(x)=\sum_y\fXY,$$
so
$$f_X(0)=f_{X,Y}(0,0)+f_{X,Y}(0,1)+f_{X,Y}(0,2)+\cdots.$$
\end{remark}
