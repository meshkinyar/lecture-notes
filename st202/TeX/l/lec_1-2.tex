% ! TeX root = ..\main.tex
\lecture{1}{Lecture 2}{Wed 29 Sep 2021 10:00}{}

Let $(\psi, \mathcal G)$ be a measurable space, with $\omega\in \psi$ and $A\in \mathcal G$. Consider 
\begin{align*}
   m(A)=\mathbf 1_A(\omega)
   = \begin{cases}
   1, \text{ if $w\in A$} \\
   0, \text{ if $w\not \in A$}.
   \end{cases}
\end{align*}

\begin{ex}Check that this is a measure! (on the problem set).
\end{ex}

\subsection{The Probability Measure}

\begin{definition}
Consider the measurable space 
$(\Omega,\mathcal F).$
Define $(\Omega,\mathcal F,P)$ as a \textbf{probability space.} The function $P$ is a \textbf{probability measure} that satisfies $P(A)\in[0,1]$ for all $A\in \mathcal F$ and $P(\Omega)=1.$
\end{definition}
\vskip.1in
Since $P$ is a measure,
\begin{itemize}
    \item $P(A)\geq 0$ for all $A\in \mathcal F,$
    \item $P(\varnothing)=0,$
    \item $P(A\cup B)=P(A)+P(B)$ \textit{if $A\cap B=\varnothing$ (mutually exclusive).}
\end{itemize}

\vskip.1cm

In general, if $A,B\in \mathcal F$ do we have $A\cap B\in \mathcal F?$ Observe that

$$(A\cap B)^c=A^c\cup B^c.$$ So yes! We do.
\vskip.1in
In general, if $A_1,A_2,A_3,\dots\subseteq \Omega$ are mutually exclusive, then

$$
P\left(\bigcup^\infty_{i=1} A_i\right)=\sum^\infty_{i=1}P(A_i).
$$

\subsubsection{Basic Properties of Probability Measures}

\begin{enumerate}[i.]
    \item $P(A^c)=1-P(A)$
    \item If $A\subseteq B$, then $P(B\setminus A)=P(B)-P(A)$
    \item $P(A\cup B)=P(A)+P(B)-P(A\cap B)$
\end{enumerate}
Example proof (the remaining proofs are left as an exercise):
\begin{proof}
\vskip.1in

\begin{enumerate}[i.]
    \item $A,A^c$ are disjoint, and thus $A\cup A^c=\Omega$, but $P(A\cup A^c)=P(A)+P(A^c).$
\end{enumerate}
\end{proof}

\begin{corollary}{\relax}
If $A\subseteq B$, then $P(A)\leq P(B).$
\end{corollary}

\subsubsection{General Addition Rule:}

\begin{align*}
    P\left(\bigcup_{i=1}^n A_i\right)&=\sum_{i=1}^n P(A_i)-\sum^n_{i,j=1,\ i<j} P(A_i\cap A_j) \\
    &+\sum_{i<j<k}P(A_i\cap A_j\cap A_k) \\
    &- \dots \\
    &+(-1)^{n+1}P(A_1\cap A_2\cap \cdots \cap A_n).
\end{align*}

\subsection{More Properties of Probability Measures}

\begin{theorem}[Boole's Inequality]
If $(\Omega,\mathcal F,P)$ is a probability space and $A_1,A_2,A_3,\dots\in \mathcal F$, then: $$P\left(\bigcup^\infty_{i=1}A_i\right)\leq\sum_{i=1}^\infty P(A_i).$$
\end{theorem}

\begin{proof}
Define 

\begin{align*}
    &B_1=A_1\\
    &B_2= A_2\setminus B_1 \\
    &B_3= A_3 \setminus (B_1\cup B_2) \\
    &\vdots \\
    &B_i=A_i\setminus(B_1\cup \dots \cup B_{i-1}).
\end{align*}

    Then $B_1,B_2,B_3,\dots\in \mathcal F$ (confirm this!) They are \textit{disjoint,} and $\bigcup_{i=1}^\infty B_i=\bigcup^\infty_{i=1}A_i.$ So,
    $$P\left(\bigcup_{i=1}^\infty A_i\right)=P\left(\bigcup_{i=1}^\infty B_i\right)=\sum^\infty_{i=1}P(B_i)\leq \sum^\infty_{i=1}P(A_i).
    $$
\end{proof}
\begin{prop}
		
If $A_1,A_2,A_3,\dots$ is an increasing sequence of sets $A_1\subseteq A_2\subseteq A_3\subseteq \dots$, then $\lim_{n\to \infty} P(A_n)=P\left(\bigcup_{i=1}^\infty A_i\right).$ 
\end{prop}
The following figure may come in handy:
\incfig[0.5]{w1_l2_fig1}{Visual representation of $A_1,A_2,\ldots$}
\begin{proof}
Define
\begin{align*}
    &B_1=A_1 \\
    &B_2=A_2\setminus A_1 \\
    &\vdots \\
    &B_i=A_i\setminus A_{i-1}\\
    &\vdots 
\end{align*}


Note that these events are mutually exclusive, and so $A_n=\bigcup^n_{i=1} B_i$. Moreover, $\bigcup_{i=1}^\infty B_i=\bigcup^\infty_{i=1}A_i$. Hence,

\begin{align*}
    \lim_{n\to \infty}P(A_n)&=
    \lim_{n\to \infty}P\left(\bigcup^n_{i=1} B_i\right) \\
    &=\lim \sum^n_{i=1}P(B_i) \\
    &= P\left(\bigcup^\infty_{i=1}B_i\right) \\
    &  = P\left(\bigcup^\infty_{i=1}A_i\right).
\end{align*}
\end{proof}
We will use $P(A)=\frac{|A|}{|\Omega|},$ for $A\in \mathcal F$, with assumptions that the each event is equally likely, and that the sample space is finite.

\subsection{Sample Problems}

\begin{enumerate}[1)]
    \item \textbf{Lottery}: choose 6 numbers from $\{1,2,\dots,59\}$. What is the probability of matching 6 numbers?
    
    \item \textbf{Birthdays}: 100 people in this lecture. What is the probability that at least two share a birthday?
\end{enumerate}

\begin{note}
		{Read how the multiplication rule applies to permutations and combinations.}
\end{note}
