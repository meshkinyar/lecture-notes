% ! TeX root = ..\main.tex
\lecture{7}{Lecture 1}{Tue 9 Nov 2021 14:00}{}

\subsection{Functions of Random Variables}

Let $X$ be a random variable, and $g:\RR\to \RR$ be a well-behaved function. We're interested in

$$Y= g(X), \quad \quad \EE(g(X))$$

When we first encountered functions of random variables, we started with the CDF, and we worked from there. But observe that
\begin{align*}
		F_Y(y)&=P(Y\leq y)\\
			  &=P(g(X)\leq y)\neq P(X\leq g^{-1}(y))
.\end{align*}
Hence, this doesn't work, e.g., for $g(x)=x^2.$

\vskip.1in

\begin{definition}
If $B\subseteq 
\RR$ and $g:\RR\to \RR$, the \textbf{inverse image} of $B$ is defined as 
$$
g^{-1}(B)=\{x\in \RR: \ g(x)\in B\}.
$$
\end{definition}

\begin{eg}
If $g(x)=x^2,$
\begin{align*}
    g^{-1}(\{4\})&=\{-2,2\} \\
    g^{-1}([0,1])&=[-1,1]
\end{align*}
\end{eg}

\vskip.1in 
\begin{eg}
	For a random variable $Y$ and some set $B$,
\begin{align*}
    P(Y\in B) &= P(g(X)\in B) \\
    &= P(\{\omega\in \Omega: \ g(X(\omega)) \in B\}) \\
    &=P(\{\omega \in \Omega: \ X(\omega)\in g^{-1}(B) \}) \\
    &=P(X\in g^{-1}(B))
\end{align*}
\end{eg}

\begin{remark}
Note that
\begin{align*}
    F_Y(y)&=P(Y\leq y)\\
    &=P(Y\in(-\infty,y])\\
    &=P(X\in g^{-1}((-\infty,y]) \\
    &=\begin{dcases}
    \sum_{x:g(x)\leq y} \quad &\text{(discrete)} \\
    \int_{x:g(x)\leq y}f_X(x) \ dx \quad &\text{(continuous).}
    \end{dcases}
\end{align*}
Further,
\begin{align*}
    f_Y(y)=\ldots=\sum_{x:g(x)=y}f_X(x).
\end{align*}
\end{remark}

\begin{eg}
Let $X$ be a continuous random variable and $Y=g(X)=X^2.$ For $y\geq 0$:
\begin{align*}
    F_Y(y)&=P(Y\leq y) \\
    &=P(X^2\leq y) \\
    &=P(-\sqrt y \leq X \leq \sqrt y \\
    &= F_X(\sqrt y)-F_X(-\sqrt y) \\
    &\implies f_Y(y)&=\der{y}F_Y(y) \\
    &=\begin{cases}
    \frac{1}{2\sqrt y}(f_X(\sqrt y) +f_X(-\sqrt y), \quad &y\geq 0 \\
    0, \quad &y<0. 
    \end{cases}
\end{align*}

If $X\sim \Normal(0,1),$ then
\begin{align*}
    f_y(y)&=\frac{1}{2\sqrt y}\left(\frac{1}{\sqrt{2\pi}}e^{\frac{-(\sqrt y)^2}{2}}+\ldots\right) \\
    &=\frac{1}{\sqrt{2\pi}} y^{-\frac{1}{2}}e^{-\frac{y}{2}}, \quad y\geq 0.\\
    &= \frac{(1/2)^2}{\sqrt \pi}y^{\frac{1}{2}-1}e^{-\frac{1}{2}y}.
\end{align*}
Note that $\sqrt{\pi}=\Gamma(\frac12)$. Hence, $Y\sim \Gamma\left(\frac{1}{2},\frac{1}{2}\right).$
\end{eg}

\vskip.1in

\subsubsection{Monotonicity}

\begin{definition}
		A function is \textbf{monotone} if it is strictly increasing or strictly decreasing.
\end{definition}

\begin{remark}
If a function is monotone increasing, then
\[
y\in(c,d)\iff x\in (a,b),
.\] 
and hence, $g^{-1}((c,d))=(a,b).$


\incfig{w7_l1_fig1}{A monotone increasing function}

Similarly, if a function is monotone decreasing, then
\[
y\in (c,d) \iff x\in (a,b),
.\] 
and hence, $g^{-1}((c,d))= (a,b)$

\incfig{w7_l1_fig2}{A monotone decreasing function}

In general, if $g$ is monotone (increasing or decreasing),

\begin{align*}
    g^{-1}((-\infty,y])
    &=\{x\in \RR: \ g(x)\in(-\infty,y]\} \\
    &=\begin{cases}
    (-\infty,g^{-1}(y)], \quad &g \text{ is increasing}\\
    [g^{-1}(y),\infty), \quad &g \text{ is increasing}.
    \end{cases}
\end{align*}
\begin{align*}
    F_Y(y)&=P(X\in g^{-1}((-\infty,y])) \\
    &=\begin{cases}
    P(X\in (-\infty,g^{-1}(y)]), \quad & g \uparrow \\
    P(X\in (g^{-1}(y),\infty]), \quad & g \downarrow 
    \end{cases} \\
    &=\begin{cases}
    F_X(g^{-1}(y)) \quad  & g \uparrow \\
    1-F_X(g^{-1}(y)-), \quad & g \downarrow 
    \end{cases}
\end{align*}
\end{remark}
\begin{note}
$F_X(x-)=\lim_{h\downarrow 0}F_X(x-h)=P(X<x).$
\end{note}

\begin{remark}
If $X$ is continuous, then
\begin{align*}
    f_Y(y)&=
    \begin{dcases}
    \der{y}F_X(g^{-1}(y)), \quad & g \ \uparrow \\
    \der{y}(1-F_X(g^{-1}(y))), \quad & g \ \downarrow
    \end{dcases} \\
    &=\begin{dcases}
    \left(\der{y}g^{-1}(y)\right)f_X(g^{-1}(y)), \quad & g \ \uparrow \\
    \left(-\der{y}g^{-1}(y)\right)f_X(g^{-1}(y)), \quad & g \ \downarrow 
    \end{dcases}\\
    &=f_X(g^{-1}(y))\left|\der{y}g^{-1}(y)\right|, \quad g \ \uparrow \text{ or } \downarrow.
\end{align*}

\end{remark}
\begin{eg}
Let $Y=e^X.$ Note that
$$
g(x)=e^x\iff g^{-1}(y)=\log y,
$$
so
\begin{align*}
    f_Y(y)&=f_X(g^{-1}(y))\left|\der{y}g^{-1}(y)\right| \\
    &= f_X(\log y) \left|\frac{1}{y}\right| \\
	&=f_X(\log y)\frac{1}{y}, \quad y\geq 0.
\end{align*}
If we define $y=g(x),$ $x=g^{-1}(y),$ we can write
$$
\boxed{f_Y(y)=f_X(x)\left|\der[x]{y}\right|}
$$
\end{eg}
