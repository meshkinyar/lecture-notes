% ! TeX root = ..\main.tex

\lecture{21}{Lecture 2}{Thu 24 Mar 10:00 2022}{}

\subsection{The Hat and the Annihilator}

Let $\YY=\XX\boldsymbol \theta +\boldsymbol \epsilon$ be a linear model. Recall that $\EE(\bepsilon)=0$, and $\Var(\boldsymbol\epsilon)-\sigma^2\boldsymbol I_n$. Note that 
 \begin{align*}
	 S(b)=(\YY-\hat{\YY})^T(\YY-\hat{\YY})=\sum^n_{i=1}(Y_i-\hat{Y}_i)^2
.\end{align*}
Then $\hat{\boldsymbol{{\beta}}}=(\XX^T\XX)^{-1}\XX^T\YY$ minimizes $S(b)$, and
$$\YY=\XX\hat{\boldsymbol{{\beta}}}=\underbrace{\XX(\XX^T\XX)^{-1}\XX^T}_{\mathbf{H}, \text{ the hat matrix.}}\YY=\mathbf {H}\YY.$$

The residuals are
\begin{align*}
		\hat{\boldsymbol{{\epsilon}}}&=\YY-\XX\hat{\boldsymbol{{\beta}}}\\
									 &=\YY-\hat{\YY}\\
									 &=\YY-\mathbf{H}\YY \\
									 &=\hspace{-2.5em}\underbrace{(\mathbf I_n-\mathbf{H})}_{\mathbf A,\text{ the annihilator matrix}} \hspace{-2.5em}\times \YY\\ 
									 &=\mathbf A \YY
.\end{align*} 

Note that $\HH$ and $\mathbf A$ are symmetric and idempotent.

\begin{definition}
	An \textbf{idempotent} matrix $\mathbf A$ satisfies the property $\mathbf A=\mathbf A^2$. 
\end{definition}

\subsection{Linear Models I}
\[
		\mathbf 1_n=\underset{\substack{\\ (n\times 1)}}{\begin{pmatrix} 1 \\ 1 \\ \vdots \\ 1, \end{pmatrix}}, \quad \mathbf e_1= \underset{\substack{\\ (p+1)\times 1}}{\begin{pmatrix} 1 \\ 0 \\ \vdots \\ 0 \end{pmatrix}} 
.\]
Moreover,
\[\XX=
		\underset{\substack{\\ \\ n\times(p+1)}}{\begin{pmatrix} 1 & X_{1,1} &\cdots & X_{1,p} \\ \vdots &\vdots &\ddots &\vdots \\ 1 & X_{n,1} &\cdots X_{n,p} \end{pmatrix}}
.\] 
Then

\begin{align*}
	&\mathbf 1^T\YY=\sum^n_{i=1}Y_i, \quad \XX \mathbf e_1=1
.\end{align*}

Moreover,

\begin{align*}
	\mathbf 1^T \hat{\YY} 
	&= (\XX\mathbf e_1)^T \XX(\XX^T\XX)^{-1} \XX^T\YY \\
	&= \mathbf e_1^T \XX^T\XX(\XX^T\XX)^{-1}\XX^T\YY\\
	&= \mathbf 1^T \YY
.\end{align*}
Or, 
\begin{align*}
	\sum_{i=1}^{n} \hat{Y_i}=\sum_{i=1}^nY_i \iff \sum^n_{i=1} \left(Y_i-\hat{Y_i} \right) =0
,\end{align*}
i.e., the residuals sum to zero. Moreover,

\begin{align*}
	\frac{1}{n}\mathbf 1^T\XX 
	&=\frac{1}{n}\left( n\sum^n_{i=1}X_{i,1} \cdots \sum^n_{i=1}X_{i,p}  \right) \\
	&= (1\times  \bar{X}_{\cdot 1}\ldots \bar{X}_{\cdot p}) \\
	&= \bar{\mathbf X}^T 
.\end{align*}

Then 
\begin{align*}
	\hat{\YY} &=\bar{\XX}\hat{\bbeta} \\
	&= \frac{1}{n} \mathbf 1^T \bar{\XX} \hat{\bbeta}\\
	&= \frac{1}{n}\mathbf 1^T\XX (\XX^T\XX)^{-1}\XX^T\YY  \\
	&= \frac{1}{n} \mathbf 1^T \hat{\YY} \\
	&= \frac{1}{n}\sum^{n}_{i=1} Y_i \\
	&=\bar{Y}
.\end{align*}

This shows that the hyperplane passes through the vector $\left(\bar{\XX}, \YY \right) $.


\subsubsection{Properties }
Some properties:
\begin{enumerate}
	\item $\EE(\hat{\boldsymbol{\beta}}=\boldsymbol \beta$ (unbiased).
	\item $\Var(\hat{\boldsymbol\beta})=(\XX^T\XX)^{-1}\sigma^2$.
	\item $\EE(\hat{\boldsymbol \epsilon})=0.$
\end{enumerate}

To estimate $\sigma^2$, take

\begin{align*}
	\sum^n_{i=1}\hat{\epsilon_i^2} = \hat{\boldsymbol \epsilon}^T \hat{\boldsymbol \epsilon}=S(\hat{\boldsymbol \beta}) 
.\end{align*}

This has $n-p-1$ degrees of freedom, so
\[
	\hat{\sigma^2} = \frac{1}{n-p-1} \sum^n_{i=1}\hat{\epsilon}_i^2
.\] 
\begin{eg}
If $\YY=\XX\boldsymbol \bbeta+\boldsymbol \epsilon,$ $\boldsymbol \epsilon \sim \Normal(0,\sigma^2\mathbf I_n),$ then $\YY\sim\Normal(\XX\bbeta,\sigma^2\mathbf I_n)$. Also:
\[
	\hat{\bbeta}=(\XX^T\XX)^{-1}\XX^T\YY \quad \text{is Normal}
.\] 
because for a multivariate normal distribution, any linear transformation of the variable is also normal. Moreover,
\[
\hat{\YY=\mathbf{H}\YY} \quad \text{is Normal}
.\] 
and 
\[
\hat{\boldsymbol \epsilon} =(\mathbf I_n-\mathbb{H})\YY) \quad \text{is Normal}
.\] 
and
\[
\hat{\bepsilon} = (I_n-\HH)\YY \quad \text{is Normal}
.\] 
\end{eg}

Now, what if $Y_i$ is Bernoulli distributed? 

\begin{definition}
	The \textbf{mean function} is \[
	\mu(\mathbf x_i)=\EE(Y_i\mid \XX_i=\mathbf x_i)=\mathbf x_i^T\boldsymbol \beta
	.\] 			
\end{definition}

\begin{definition}
	Let $g(\mu(\mathbf x_i))=\mathbf x_i^T\boldsymbol\beta,$ where $g$ is the \textbf{link function}. Then
	$\mu(\mathbf x_i)=g^{-1}(\mathbf x_{i}^T\boldsymbol \beta).$
\end{definition}

\begin{note}
	The link function is the inverse of the mean function.
\end{note}

\begin{eg}[Logistic Regression]	
	Let $Y_i\mid \XX_i=\mathbf x_i\sim \Bernoulli(p(\mathbf x_i))$, where  \[
	p(\mathbf x_i)=\mu(\mathbf x_i)=\EE(Y_i\mid \XX_i=\mathbf x_i)
	.\] 
\end{eg}

Note that \[
		\underbrace{\log\left( \frac{p_i}{1-p_i}\right)}_{\in \RR}=\mathbf x_i^T\boldsymbol \beta,
\] 

so $$\mu(\xx_i^T\bbeta)=\frac{e^{\xx_i^T\bbeta}}{1+e^{\xx_i^T\bbeta}}.$$
