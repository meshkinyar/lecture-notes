% ! TeX root = ..\main.tex
\lecture{5}{Lecture 2}{Wed 28 Oct 2021 10:00}{}
\vskip.2cm
\subsection{Moment-Generating Function}

\begin{definition}
The \textbf{moment-generating function} (MGF) of a random variable $X$ is a function $M_X:\RR\to \RR^+_0$ given by 
\begin{align*}
M_X(t)=\EE(e^{tX})=\begin{dcases}
\sum_x e^{tx} f_X(x) \quad &\text{(discrete)} \\
\intR e^{tx} f_X(x) \ dx \quad &\text{(continuous),}
\end{dcases}
\end{align*}
where we require that $M_X(t)<\infty$ for all $t\in [-h,h]$ for some $h>0$ (a neighborhood of 0).
\end{definition}

\begin{remark}
Note that
$$
e^y=1+y+\frac{y^2}{2!}+\frac{y^3}{3!}+\cdots = \sum^\infty_{j=0} \frac{y^j}{j!}.
$$

And so
\begin{align*}
    M_X(t)&=\EE(e^{tX})\\
    &=\EE\left(1+tX+\frac{(tX)^2}{2!}+\cdots\right) \\
    &=\EE\left[\sum^\infty_{j=0}\frac{(tX)^j}{j!}\right] \\
    &=\sum^\infty_{j=0} \frac{t^j}{j!}\EE(X^j)\\
    &=1+t\mu'_1+\frac{t^2}{2!}\mu_2'+\frac{t^3}{3!}\mu'_3+\cdots
\end{align*}

The coefficient of $t^r$ is $\frac{\mu_r '}{r!}=\frac{\EE(X^r)}{r!}.$
\end{remark}
\vskip.1in

\begin{prop}
The $r$th derivative of $M_X(t)$ at $t=0$ is $\mu_r'.$
\end{prop}

\begin{proof}
\begin{align*}
    M_X^{(r)}(t)&=\frac{d^r}{dt^r}\mu_X(t) \\
    &=\frac{d^r}{dt^r}\left(1+t\mu_1'+\frac{t^2}{2!}\mu_2 ' +\frac{t^3}{3!}\mu_3'+\cdots\right)\\
    &=\mu_r'+t\mu'_{r+1}+\frac{t^2}{2!}\mu'_{r+2}+\cdots 
\end{align*}
This implies
$$
	\mu_X^{(r)}(0)=\mu_r'=\EE(X^r).
$$
\end{proof}

\begin{prop}
If $X,Y$ are random variables and we can find $h>0$ such that $M_X(t)=M_Y(t)$ for all $|t|<h$, i.e., $t\in (-h,h),$ then 
$$
F_X(x)=F_Y(x) \quad \text{for all } x\in \RR.
$$
\end{prop}

\begin{proof}
Omitted. 
\end{proof}

\begin{eg}
Let $X\sim \Poisson(\lambda).$ Observe that
\begin{align*}
    M_X(t)&=\EE(e^{tX})\\
    &=\sum_x e^{tx}f_X(x) \\
    &= \sum^\infty_{x=0}e^{tx}\frac{e^{-\lambda}\lambda^x}{x!} \\
    &= \sum^\infty_{x=0} \frac{e^{-\lambda}(\lambda e^t)^x}{x!} \\
    &=e^{\lambda e^t}e^{-\lambda}\sum^\infty_{x=0}\frac{e^{-\lambda e^t}(\lambda e^t)^x}{x!} \\
    &=e^{\lambda(e^t-1)}\quad \text{for } t\in \RR \\
    &=\exp(\lambda(e^t-1)) \\
    &=\exp(\lambda(e^t-1)) \\
    &=\exp(\lambda(1+t+\frac{t^2}{2}+\frac{t^3}{6}+\cdots-1\\
    &=1+\lambda(t+\frac{t^2}{2}+\cdots)+\frac{\lambda^2(t+\frac{t^2}{2}+\cdots)^2}{2}+\cdots \\
    &=1+\lambda t +\frac{\lambda t^2}{2}+\frac{\lambda^2t^2}{2} +\cdots \\
    &=1+\lambda t +\frac{\lambda t^2}{2}+(\lambda+\lambda^2)\frac{t^2}{2}+\cdots.
\end{align*}
From this, $\EE(X)=\lambda,$ and $\EE(X^2)=\lambda+\lambda^2.$ Moreover,
$$
\Var(X)=\lambda+\lambda^2-\lambda^2=\lambda.
$$

Or,
$$
M_X'=\exp(\lambda(e^t-1))\lambda e^t \implies \mu_1'=M_X'(0)=\lambda.
$$
\end{eg}

\begin{eg}
Let $Y\sim \Gamma(\alpha, \lambda).$ Then
\begin{align*}
    M_Y(t)&=\EE(e^{tY}) \\
    &=\int^\infty_0 e^{tY}\frac{\lambda^\alpha}{\Gamma(\alpha)}y^{\alpha-1}e^{-\lambda y} \ dy \\
    &=\frac{\lambda^\alpha}{(\lambda-t)^\alpha}\int^\infty_0\frac{(\lambda -t)^\alpha}{\Gamma(\alpha) y^{\alpha-1}e^{-(\lambda-t)y}} \ dy \\
    &=\left(\frac{\lambda}{\lambda-t}\right)^\alpha \\
    &=\left(1-\frac{t}{\lambda}\right)^{-\alpha}, \quad \text{for } |t|< \lambda.
\end{align*}
\end{eg}

\subsubsection{Negative Binomial Expansion}
\begin{align*}
    M_Y(t)&=\left(1-\frac{t}{\lambda}\right)^{-\alpha} \\
    &=\sum^\infty_{j=0}\binom{j+\alpha-1}{\alpha-1}\left(\frac{t}{\lambda}\right)^j 
\end{align*}
So, for example, the coefficient of $\frac{t^j}{j!}$ is $\frac{(j+\alpha-1)!}{(\alpha-1)!}\lambda^{-j}.$ Then
\begin{align*}
    \EE(Y)&=\frac{(1+\alpha-1)!}{\alpha-1)!}\lambda^{-1} \\
    &=\frac{\alpha!}{(\alpha-1)!}\frac{1}{\lambda}\\
    &=\frac{\alpha}{\lambda}.
\end{align*}

\subsection{Cumulant-Generating function}

\begin{definition}
The \textbf{cumulant-generating function} (CGF) of a random variable $X$ is $K_X(t)\ln(M_X(t)).$
\end{definition}

We can write
$$
K_X(t)=\kappa_1 t +\frac{\kappa_2}{2!}t^2+\frac{\kappa_3}{3!}t^3+\cdots
$$

The $r$th cumulant, $\kappa_r,$ is the coefficient of $\frac{t^r}{r!}$ in the power series expansion of $K_X(t)$ about 0.

\begin{eg}
Let $X\sim \Poisson(\lambda).$ Then

\begin{align*}
    K_X(t)&=\ln M_X(t) \\
    &=\ln(\exp(\lambda(e^t-1))) \\
    &=\lambda(e^t-1) \\
    &=\lambda t +\lambda \frac{t^2}{2} +\lambda \frac{t^3}{3!}+\cdots.
\end{align*}
So, $\kappa_1=\kappa_2=\kappa_3=\cdots=\lambda.$
\end{eg}

\begin{prop}
If $X$ is a random variable, then
\begin{enumerate}[i.]
    \item $\kappa_1=\mu'_1=\EE(X)$
    \item $\kappa_2=\mu'_2-(\mu_1')^2=\mu_2=\Var(X)$
    \item $\kappa_3=\mu_3=\EE[(X-\EE(X))^3].$
\end{enumerate}
\end{prop}

\begin{proof}
		\phantom x
\begin{enumerate}[i.]
    \item Observe that
    \begin{align*}
		K_X(t)&=\ln(M_X(t)) \\
			\implies K_X'(t)&=\frac{1}{M_X(t)}M_X'(t) \\
			\implies \kappa_1=K_X'(0)&=\frac{1}{M_X(0)}M_X'(0)=\mu_1.
    \end{align*}
    \item 
    \begin{align*}
        &K_X''(t)=\left(\frac{M_X'(t)}{M_X(t)}\right)' = \frac{M''_X(t)M_X(t)-(M_X'(t))^2}{(M_X(t))^2}\\
        &\implies \kappa_2=K''_X(0)=\mu_2'-(\mu_1')^2.
    \end{align*}
    \item Left as an exercise.
\end{enumerate}
\end{proof}
